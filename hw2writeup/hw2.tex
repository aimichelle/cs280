\documentclass[11pt]{article}
\usepackage{amsmath,textcomp,amssymb,graphicx,alltt}
\usepackage[top=1in, bottom=1in, left=.5in, right=.5in]{geometry}

\def\Name{Allen Tang, Michelle Nguyen}  % Your name

\title{CS280 -- Spring 2015 -- Homework 2}
\author{\Name}
\markboth{CS280 -- Spring 2015 Homework 2 \Name}{CS174 -- Spring 2015 Homework 2 \Name}

\pagestyle{myheadings}
\begin{document}
\maketitle

\section*{1.1 Q1}
THE GRAPHIC IS 'Question1.1.png'
\newpage
\section*{1.2 Q2}
\begin{itemize}
\item[a)]
Spatial pyramids give a regional "density" to the image. Some digits such as $4$ and $9$ may be classified similarly if we just use raw pixels, but if we use spatial pyramids, we are able to detect that many $4$'s have an opening on the top of the digit, where $9$'s do not. This difference can be captured by a spatial pyramid.
\item[b)]
THE GRAPHIC IS 'Total Sum c4.png'
THE GRAPHIC IS 'Total Sum c7.png'
We get roughly $3\%$ difference in classification accuracy on the test set, which is pretty significant. 
\end{itemize}
\newpage


\newpage
\section*{1.3 Q3}	
\begin{itemize}
\item[a)]
Gradient orientations should help because they capture the relative ratios of gradient directions, which is invariant to slant (kind of like a shear).
\item[b)]
We get a huge increase in accuracy, about $10\%$
\end{itemize}
\newpage
\section*{4.}

\end{document}